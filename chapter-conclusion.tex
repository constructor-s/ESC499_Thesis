\chapter{Conclusion}

Glaucoma is often referred to as the ``silent thief of vision'' because its irreversible progression is unnoticeable without visual field testing until the very late stages. Therefore, medical doctors need to have the best information available as early as possible to determine the risk and benefits of different treatment approaches. It is widely agreed that 5--6 visual fields, or approximately 2 years, are required before a clinically useful progression rate statistic can be determined. 

The motivation behind the current work is to investigate if a machine learning based approach may yield better clinical information for treatment decisions. Specifically, the Rotterdam public dataset is used to evaluate 5 different machine learning algorithms of different complexities, including linear models (ridge and Huber regression), neural network models (\ac{MLP} and \ac{CNN}), and \iac{RNN} approach (\acs{CNN}+\acs{LSTM}). The motivation behind the \ac{CNN} architecture is its ability to extract spatial features, while the motivation for the \ac{LSTM} architecture is its ability to model sequential temporal inputs. 

The different approaches are evaluated on two tasks: (1) predicting future \ac{MD} using 3 inputs fields and (2) predicting future point-wise field thresholds using 3 input fields. It is found that all machine learning methods produce similar results with better predictions in both tasks than using than linear extrapolation, but do not outperform simple repetition of the last observed value. This is due to the fact that the Rotterdam dataset consists primarily of patients who are not progressing, and therefore a non-progressing prediction performed quite well. Therefore, future studies should be based on a more comprehensive dataset, including a significant number of moderately to rapidly progressing patients. This also illustrates the challenges associated with achieving both high sensitivity and specificity in the glaucoma prediction tasks in general. 
